\chapter{Fazit}
\lhead{Kapitel 6: \emph{Fazit}}

Das Projekt wurde exemplarisch einigen Studierenden vorgestellt. Einige der Studierende hatten bereits Vorkenntnisse in der Lichttechnik oder auch in der Elektrotechnik. Die Studierenden mit Vorkenntnisse hatten bereits ein grobes Verständnis, wie ein Datenprotokoll aufgebaut ist und wie die Daten decodiert werden. Der Großteil der Studierenden hatten jedoch keine Vorkenntnisse in der Lichttechnik oder von Steuerprotokollen.

Die Studierenden haben einen guten Eindruck vom Protokoll erhalten, und konnten auch schnell die Limitierungen des Protokolls erkennen. So wurde den meisten schnell bewusst, wie die DMX Kanäle für die Steuerung von Leuchten verwendet werden können. Auch war den Studierenden schnell erkenntlich, warum ein Kanal einen Wertebereich von 0 bis 255 besitzt.

Auch die Studierenden, welche bereits Vorkenntnisse hatten, haben sich für die genauen Zeitabstände innerhalb des Protokolls interessiert. Die analysierten Zeitabstände haben kleine Messunsicherheiten, für dieses Projekt stellt das jedoch kein Problem da.

Der Formfaktor des Projektes ist ebenfalls gut, da er leicht zu transportieren ist, aber auch groß und robust genug ist, um es anderen Studierenden zu demonstrieren.

Es können zwischen verschiedenen Bildschirmansichten gewechselt werden, um verschiedene Aspekte des Protokolls genauer zu demonstrieren. Dadurch ist es noch einfacher das Protokoll zu verstehen.

Das gesamte Projekt ist öffentlich \cite{githubDmxAnalyzer} einsehbar, und gut dokumentiert, damit andere Personen die Möglichkeit haben weiter auf diesem Projekt aufzubauen.

