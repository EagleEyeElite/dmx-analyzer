\thispagestyle{plain} % No headers

{\huge\textbf{Abstrakt}}
\vspace{1cm}

In der Lichttechnik haben sich seit den 80er-Jahren verschiedene Standards durchgesetzt. Je nach Anwendungsfall werden verschiedene Protokolle präferiert. Die Protokolle DMX, Dali, und ZigBee haben sich besonders in der Industrie etabliert. Im Hobbybereich findet auch vermehrt das MIDI Protokoll Anwendung.

In der Bachelorarbeit werden die Stärken und Schwächen der Protokolle miteinander verglichen. Dabei werden Faktoren wie die Geschwindigkeit, Nachrüstbarkeit und Benutzerfreundlichkeit näher untersucht.

In der Bühnentechnik hat sich das DMX Protokoll als Industriestandard bewährt. Der Aufbau des Protokolls ermöglicht es, Datenpakete einfach mitzuschneiden. In der Bachelorarbeit wird für die Lehre ein Projekt gebaut, um Studierenden das DMX Protokoll anschaulich zu erklären. Dabei wird das DMX Protokoll sowohl auf der Anwenderschicht intuitiv erklärt, als auch auf der physischen Übertragungsschicht aufgeschlüsselt.