\thispagestyle{plain} % No headers

{\huge\textbf{Abstract}}
\vspace{1cm}

In der Lichttechnik haben sich seit den 80er-Jahren verschiedene Standards durchgesetzt. Je nach Anwendungsfall werden verschiedene Protokolle präferiert. Die Protokolle DMX, Dali, und seit Neuerem auch Zigbee haben sich besonders in der Industrie etabliert. Im Hobbybereich findet auch vermehrt das MIDI Protokoll Anwendung.

In der Bachelorarbeit werden die Stärken und Schwächen der Protokolle miteinander verglichen. Dabei werden Faktoren wie die Geschwindigkeit, Nachrüstbarkeit, Fehlertoleranz und easy of use näher untersucht.

In Bühnentechnik hat sich DMX als Industriestandard bewährt. Der Aufbau des Protokolls ermöglicht es, Datenpakete einfach mitzuschneiden. Für die Lehre wird ein Projekt gebaut, welches das Protokoll auf Bit ebene veranschaulicht.

