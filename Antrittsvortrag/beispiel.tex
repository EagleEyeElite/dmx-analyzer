\documentclass[Nike]{tuberlinbeamer}

\usepackage[ngerman]{babel}  % 'babel' muss geladen werden
\usepackage[utf8]{inputenc}  % optional, aber empfehlenswert

% Die ueblichen Angaben
\title{\LaTeX-Folien im Corporate Design der TU Berlin}
\subtitle{Untertitel}
\author[Kurzname]{Vorname Nachname}
\institute{Technische Universität Berlin}

% Eigenes Logo einfuegen:
\renewcommand{\pathtomylogo}{meinlogo}

\begin{document}

\begin{frame}
\maketitle
\end{frame}


\begin{frame}
\tableofcontents
\end{frame}

\section{Einleitung}

\begin{frame}[fragile]{Verwendung der \texttt{tuberlinbeamer}-Klasse}
Es folgen demnächst ein paar Folien zur Verwendung dieser Dokumentklasse.
\begin{itemize}
\item Kenntnis der \emph{beamer}-Klasse ist von Vorteil
\end{itemize}
\end{frame}

\section{Zusammenfassung}

\begin{frame}{ToDo}
\begin{itemize}
\item \emph{ToDo} schreiben
\item \emph{ToDo} abarbeiten
\end{itemize}
\end{frame}

\end{document}
